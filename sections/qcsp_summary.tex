\documentclass[../main.tex]{subfiles}

\makeatletter
\@ifundefined{fromRoot}{%
  \newcommand{\fromRoot}[1]{../#1}
  
  \usepackage{xr}
  \externaldocument{../main}
}{}

\def\input@path{{\subfix{../}}}
%or: \def\input@path{{/path/to/folder/}{/path/to/other/folder/}}
\makeatother

\graphicspath{{\subfix{../}}}

\hypersetup{
    pdfauthor   = {Camille MONIÈRE},
    pdftitle    = {Th\`{e}se (Présentation: QCSP, un résumé)},
    pdfsubject  = {Th\`{e}se (Présentation: QCSP, un résumé)},
%    pdfkeywords = {mots-cl\'{e}s},
}

\begin{document}

\section{Système de communication \acrshort{qcsp}}

\subsection{Modèle}

\begin{frame}{\subsecname}
  \begin{center}
    \textcolor{RoyalBlue}{TODO, image de la chaine, et le canal théorique}
  \end{center}
\end{frame}

\subsection{Émission}

\begin{frame}{\subsecname}
  \begin{center}
    \textcolor{RoyalBlue}{TODO, les trois étape, et un filtre}
  \end{center}
\end{frame}

\subsection{Détection}

\begin{frame}{\subsecname : {Principe}}
  \begin{center}
    \textcolor{RoyalBlue}{TODO}
  \end{center}
\end{frame}


\begin{frame}{\subsecname : {animation CCSK demod}}
  \begin{center}
    \textcolor{RoyalBlue}{TODO}
  \end{center}
\end{frame}

\begin{frame}{\subsecname : {Problèmes Temps Fréquence}}
  \begin{center}
    \textcolor{RoyalBlue}{TODO}
  \end{center}
\end{frame}


\begin{frame}{\subsecname : {La grille temps fréquences}}
  \begin{center}
    \textcolor{RoyalBlue}{TODO}
  \end{center}
\end{frame}

\subsection{Synchronisation}

\begin{frame}{\subsecname}
  \begin{center}
    \textcolor{RoyalBlue}{TODO --- En fonction du temps, détails ou pas}
  \end{center}
\end{frame}

\subsection{Décodage}

\begin{frame}{\subsecname}
  \begin{center}
    \textcolor{RoyalBlue}{TODO --- Merci CCSK et hop}
  \end{center}
\end{frame}

\end{document}
