\documentclass[../main.tex]{subfiles}

\makeatletter
\@ifundefined{fromRoot}{%
  \newcommand{\fromRoot}[1]{../#1}
  
 % \usepackage{xr}
  % \externaldocument{../main}
}{}

\def\input@path{{\subfix{../}}}
%or: \def\input@path{{/path/to/folder/}{/path/to/other/folder/}}
\makeatother

\graphicspath{
  {\subfix{../}}
  {\subfix{./figures}}
  {\subfix{../figures}}
  {\subfix{./figures/logos-thesis/}}
  {\subfix{../figures/logos-thesis/}}
  {\subfix{./figures/rtexps-pics/}}
  {\subfix{../figures/rtexps-pics/}}
}

\hypersetup{
    pdfauthor   = {Camille MONIÈRE},
    pdftitle    = {Th\`{e}se (Présentation: conclusion)},
    pdfsubject  = {Th\`{e}se (Présentation: conclusion)},
%    pdfkeywords = {mots-cl\'{e}s},
}

\begin{document}

\section{Conclusion et Perpectives}

\subsection{Synthèse}

\begin{frame}{\subsecname}
  \begin{columns}
    \begin{column}{.7 \linewidth}
      Présenté aujourd'hui :
      \begin{ctrlitemize}{.2 em} \small
        \item Améliorations algorithmiques, en termes de robustesse et d'efficacité ;
        %, et atteint toujours des SNRs inférieurs à $-10$ dB, avec $20 \%$ de ressource spectrale de moins comparé aux protocoles à base de préambules \cite{saiedThesis2022}).
        \item Mise au point d'un démonstrateur temps réel ;
        \item Campagnes d'expérimentations grandeurs nature.
      \end{ctrlitemize} \vspace{1 em}

      Non abordé dans cette présentation, mais également étudié :
      \begin{ctrlitemize}{.2 em} \small
        \item Émetteur temps-réel ;
        \item Modèle en virgule fixe du détecteur ;
        \item Description matérielle du détecteur en cours de finalisation.
      \end{ctrlitemize}
    \end{column}
    \begin{column}{.3 \linewidth}
      \includegraphics[width = \linewidth]{checklist-1622517_1920.png}
    \end{column}
  \end{columns}
  \blfootnote{\textcite{saiedThesis2022}}
\end{frame}

\subsection{Contributions}
\begin{frame}
  \frametitle{\subsecname}
  \scriptsize
  \begin{refsection}
    \nocite{*}
    Article soumis à une revue internationale

    \vspace{.5 em}

    \renewcommand*{\bibfont}{\rikiki}
    \printbibliography[keyword=ownpub, keyword=inter, type=article, heading=none,]

    \vspace{1 em}

    Papiers en conférences internationales

    \vspace{.5 em}

    \renewcommand*{\bibfont}{\rikiki}
    \printbibliography[keyword=ownpub, keyword=inter, type=inproceedings, heading=none,]

    \vspace{1 em}

    Papiers en conférences nationales

    \vspace{.5 em}

    \renewcommand*{\bibfont}{\rikiki}
    \printbibliography[keyword=ownpub, keyword=nationale, type=inproceedings, heading=none,]
    \printbibliography[keyword=ownpub, keyword=nationale, type=misc, heading=none,]

    \vspace{1 em}
    Intervention dans des colloques nationaux

    \vspace{.5 em}

    \renewcommand*{\bibfont}{\rikiki}
    \printbibliography[keyword=ownpub, keyword=nationale, type=unpublished, heading=none,]

  \end{refsection}
  \note{
    \textbf{JSA -> Q1} \\
    \textbf{DASIP -> BEST PAPER AWARD}
  }
\end{frame}

\subsection{Perspectives futures}

\begin{frame}{\subsecname}
  \begin{columns}
    \begin{column}{.3 \linewidth}
      \hfill \includegraphics[width = .35\linewidth]{future_sp.png} \\
      \includegraphics[width = \linewidth]{steps_futur.jpg}
    \end{column}
    \begin{column}{.7 \linewidth}
      {\centering \underline{\textbf{Court terme}}\par}
      \begin{ctrlitemize}{.2 em} \small
        \item Modèle en virgule fixe validé et en cours d'implantation sur puce FPGA ;
        \item Intégration complète du détecteur matériel dans un module de radio logicielle.
      \end{ctrlitemize}

      {\centering \underline{\textbf{Long terme}}\par}
      \begin{ctrlitemize}{.2 em} \small
        \item Extensions :
        \begin{itemize}
          \item QCSP en contexte ``multi-utilisateurs'' ;
          \item Pertinence pour l'usage spatial ;
          \item Étude des couches OSI supérieures.
        \end{itemize}
        \item Domaines applicatifs variés : Maritime, terrestre, aérien, spatial \dots
      \end{ctrlitemize}
    \end{column}
  \end{columns}
  \note{
    \textit{Tournure :} en discutant avec des collègues 

    pour les extensions :

    \begin{itemize}
      \item Résilience au Doppler par conception\\
      \item Possibilité d'une plateforme SoPC
    \end{itemize}


    pour les domaines :

    \begin{itemize}
      \item Bouées autonomes ;
      \item IoT spatial ou terrestre ;
      \item paquets cours pour flottes de drones
    \end{itemize}
  }
\end{frame}


\end{document}
